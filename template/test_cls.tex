\documentclass{mai_book}

\defaultfontfeatures{Mapping=tex-text}
\setdefaultlanguage{russian}

%\clearpage
%\setcounter{page}{7} % ВОТ ТУТ ЗАДАТЬ СТРАНИЦУ
% Эти счетчики достаточно задать один раз, обновляются дальше сами


\begin{document}



\section{Секция}

\begin{namedlisting}{Name}
Text
\end{namedlisting}
 
\begin{mynotice}
Тут замечание.
\end{mynotice}
 
 
\begin{determ}[Lol]
тут что-то определить
\end{determ}

\begin{thm}[A Great Theorem]
Something Great
\end{thm}
\section{Секция}

\begin{sled}
Lol
\end{sled}

\begin{bezpodpisi}
Lol
\end{bezpodpisi}

\begin{thm}
Lol
\end{thm}


\begin{lstlisting}[caption=My Code]
int main(int argc, char **argv) {
    return 0;
}
\end{lstlisting}


\begin{mynotice}
Тут замечание.\footnote{Lol}
\end{mynotice}
В этой главе — вероятно самой «информатической» в нашей книге —
мы введем некоторые алгоритмические понятия и опишем несколько
получисленных алгоритмов, необходимых для реализации алгебраиче­
ских методов.
Точное определение алгоритмических границ, в которых будут из­
\begin{determ}
В этой главе — вероятно самой «информатической» в нашей книге —
мы введем некоторые алгоритмические понятия и опишем несколько
\hspace*{7pt} получисленных алгоритмов, необходимых для реализации алгебраиче­
ских методов.
Точное определение алгоритмических границ, в которых будут из­
\end{determ}
укоризненным способом семантику построений, используемых в алго­
ритмах 1 .
После изложения общих направлений\footnote{Лол Кек Чебурек!} мы приступим к исследова­
нию первого фундаментального алгоритма этой книги: дихотомиче­
ского алгоритма возведения в степень. Для этого будут использоваться
два метода построения алгоритмов. Один очень близок к математиче­
ским рассуждениям, другой — чисто алгоритмический. При этом у нас
появится возможность открыть первый рекурсивный алгоритм, пред­
ставленный в данной работе.
Начав с первого примера алгоритма, мы перейдем к изучению на­
чальных понятий программирования на языке Ада. И здесь тоже речь
пойдет не об изложении полного курса языка Ада, а только о пред-

\pagebreak
ставлении нескольких ключевых понятий, необходимых для понимания
приведенных программ.
Вообще говоря, все понятия будут представлены (по мере надоб­
ности) на примерах, во избежание их формального описания (по по­
следнему вопросу лучше обратиться к справочному изданию: Reference
manual for the Ada\footnote{Heh, ADA} programming language 1 ).
И наконец, завершит эту главу разработка совершенно неэффек­
тивного метода вычисления определителей целочисленных матриц. Об­
суждаемый метод, основанный на математическом определении детер­
минанта, позволяет наилучшим образом использовать все возможности
современных компьютеров и получить идеальное приближение к беско­
нечности на машине! Кроме того, обсуждение ситуации позволит нам
несколько пополнить представление о программировании на языке Ада.

\section{Введение в алгоритмику}

Прежде чем приступить к описанию любого алгоритма, нужно ого­
ворить общепринятые соглашения, используемые в работе. В инфор­
матике чаще, чем в любой другой дисциплине, встречается взаимное
непонимание между пользователем и разработчиком новых программ.
Поэтому так важно сформулировать аксиоматику, позволяющую связ­
но рассуждать об алгоритмах.
Множество аксиом, которое мы собираемся сейчас описать, извест­
но как аксиоматика Хоара [86], в честь исследователя, который первым
формализовал семантику структур и действий последовательной алго-
ритмики.

\subsection{Терминология и обозначения}

А л г о р и т м — это последовательность более или менее элементарных
действий, которая позволяет пошагово решить поставленную задачу за
конечное время. Можно формально по индукции определить, что та­
кое алгоритм, нижеследующим образом. Цель этого введения в алго­
ритмику — изучение и представление всех терминов, фигурирующих в
предыдущем определении (стиль определения, который в информатике
называется гр а м м а ти к ой ).
Можно описать алгоритм в терминах состояний системы: состояние
системы до применения алгоритма, состояние той же системы после

\pagebreak

применения алгоритма. Эти два состояния образуют то, что называет­
ся сп ец иф икац ией задачи. Таким же образом можно, описывая поэ


\end{document} 