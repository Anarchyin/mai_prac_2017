\documentclass{mai_book}
\defaultfontfeatures{Mapping=tex-text}
\setmainfont{DejaVuSerif}
\setdefaultlanguage{russian}
\usepackage{amsmath}
\usepackage{floatflt}
\usepackage{wrapfig}
\begin{document}
\begin{center}
\parbox{11cm}
{
\hspace{0.4cm} Наконец,  перестановка  —  первоначальный  объект  нашего
изучения — полностью исчезла, а все эти результаты могут при­
меняться в более общих рамках:  для нумерации биекций конеч­
ного множества.
}
\end{center}

Теперь  есть  возможность  разработать  алгоритм,  который  реали­
зует результаты предыдущего свойства. Перестановка будет предста­
влена таблицей элементов из \textit{Е},
 индексированной целыми числами, за­
ключенными между \textit{р} и \textit{q}.
 Перестановка, или скорее таблица, которая ее 
представляет, будет называться \textit{Sigma}.
  Дальнейшие построения  уточ­
няют преобразование \textit{Sigma}
 в ее  последователя в лексикографическом 
порядке.  Эта конструкция осуществляется  в три этапа и очень точно 
следует изложению предыдущего свойства.

\noindent \textbf{\large{Первая фаза}}

\begin{wraptable}{}{0.4\textwidth}
\begin{tabular}{|l}
\textit{\textbf{\begin{tabular}[c]{@{}l@{}}
Length = 1;\\ 
for(int i = q - 1; i \textgreater p; i--)\\ 
\{\\ 
\hspace{0.2cm} if(Sigma(i) \textgreater Sigma(i+1)) \\
\hspace{0.2cm} \{Length++;\}\\ 
\hspace{0.2cm} else First\_Index = i;\\ 
\}\\ 
if(Length == q - p + 1) \\ 
\{printf("No\_Successor");\}
\end{tabular}}}\\ 
\end{tabular}
\end{wraptable}
Первое, что нужно сделать, — это 
определить  конечный  максимальный 
убывающий  сегмент.  Соответствую­
щая  часть  алгоритма  очень  проста 
(смотри рядом). В этом фрагменте \textit{р} и \textit{q}
 суть границы таблицы \textit{Sigma}.
 
Цикл,  который  здесь  видим,  по­
зволяет определить \textit{First-Index}, нача­
ло финальной убывающей максималь­
ной  подпоследовательности  (величи­ну \textit{l}
  свойства  10).  Кроме того, пере­
менная \textit{Length}
сегмента,  что  позволяет  определить,  не  максимальной  ли  он  длины 
(что, как это уже было видно, означает отсутствие у перестановки по­
следователя)
 выражает длину этого сегмента,  что  позволяет  определить,  не  максимальной  ли  он  длины 
(что, как это уже было видно, означает отсутствие у перестановки по­
следователя).

\noindent \textbf{\large{Вторая фаза}}

Затем  нужно  обратить  конечный  сегмент,  определенный  в  пер­
вой фазе; алгоритм, который осуществляет эту операцию, тоже очень 
прост.

\begin{tabular}{|l}
\textit{\begin{tabular}[c]{@{}l@{}}
Ascending\_Index = First\_Index + 1\\ 
Descending\_Index = q\\ 
\textbf{while}(Ascending\_Index $<$ Descending\_Index)\\ 
\{\\ 
\hspace{0.2cm} tmp\_value $=$ Sigma(Ascending\_Index) \\
\hspace{0.2cm} Sigma(Ascending\_Index) $=$ Sigma(Descending\_Index)\\
\hspace{0.2cm} Sigma(Descending\_Index) $=$ tmp\_value \\
\hspace{0.2cm} Ascending\_Index $=$ Ascending\_Index $+$ 1\\ 
\hspace{0.2cm} Descending\_Index $=$ Descending\_Index $-$ 1\\ 
\}\\
\end{tabular}}\\
\end{tabular}
\pagebreak

Нужно заметить, что это последнее преобразование перестановки \textit{Sigma}
 сделано через транспозиции, и поэтому отсюда можно вывести 
отношение между сигнатурами начальной перестановки и полученной 
из  нее  (число  осуществленных  перемен \textit{s}  =  \textit{Length}/2,
  и,  значит, это отношение равно $(—1)^{s})$.
   
\noindent \textbf{\large{Третья фаза}}

\noindent \begin{wraptable}{i}{0.55\textwidth}
\begin{tabular}{|l|}
\hline
\textit{\textbf{\begin{tabular}[c]{@{}l@{}}
for(int i = First\_Index; i \textless q; i++)\\ 
\{\\ 
\hspace{0.3cm}if(Sigma(i) \textgreater Sigma(First\_Index)\\ 
\hspace{0.3cm}\{\\ 
\hspace{0.6cm}swap(Sigma(i),Sigma(First\_Index))\\ 
\hspace{0.3cm}\}\\ 
\}\end{tabular}}} \\ \hline
\end{tabular}
\end{wraptable}

Наконец, нужно отыскать в
только  что  обращенной  части 
таблицы  наименьший  элемент, 
превосходящий  элемент  с  ин­
дексом \textit{First-Index}
 (т.е. элемент 
с индексом \textit{l}
 из свойства 10), и
поменять их местами, что добавляет еще одну дополнительную транс­
позицию к преобразованию.

Полный алгоритм генерации биекций  формируется очень просто, 
достаточно склеить все фрагменты, которые мы только что изучили. 
Целиком это будет видно в следующем разделе, посвященном реализа­
ции всего этого.

\subsection{Настраиваемый модуль генератора биекций}
\noindent Мы закончим эту главу реализацией алгоритма перечисления биекций 
и, таким образом, построим тело цикла, спецификация которого была 
дана в  разделе 4.4. Как это было видно в предыдущем рассуждении, 
названные биекции действуют на целых интервалах, что вовсе не ли­
шает решение общности; всякая биекция между двумя конечными мно­
жествами может трактоваться как целая биекция.
Напомним все-таки, что этот пакет, кроме типа биекций между дву­
мя целыми интервалами и процедуры вычисления последующего эле­
мента, дает также наименьшую биекцию в лексикографическом поряд­
ке  (которая является и единственной  строго возрастающей  биекцией 
между двумя рассматриваемыми интервалами). Кроме того, процеду­ра \textit{Successor}
 имеет два параметра: исходная биекция и ее сигнатура, и 
эта процедура вычисляет следующую биекцию и ее сигнатуру.
\\Код Вари
\end{document}

